\documentclass[12pt]{report}
\usepackage[utf8]{vietnam}
\usepackage[left=3cm, right=3cm, top=3cm, bottom =3cm]{geometry}
\usepackage{graphicx}
\usepackage{multirow}
\usepackage{array}
\usepackage{fancyhdr}
\usepackage{fancybox}
\usepackage{hyperref}
\usepackage{float}
\usepackage{amsmath}
\usepackage{amssymb}
\usepackage{mathtools}
\usepackage{indentfirst}

\usepackage{algorithm}
\usepackage{algorithmicx}
\usepackage{algpseudocode}

% Add commands for ceil and floor
\DeclarePairedDelimiter\ceil{\lceil}{\rceil}
\DeclarePairedDelimiter\floor{\lfloor}{\rfloor}

% Add command for argmax
\DeclareMathOperator*{\argmax}{argmax}

% Change name of algorithm
\floatname{algorithm}{Thuật toán}

% set up Input, Output in the algorithm block
\algnewcommand\algorithmicinput{\textbf{Đầu vào:}}
\algnewcommand\INPUT{\item[\algorithmicinput]}
\algnewcommand\algorithmicoutput{\textbf{Đầu ra:}}
\algnewcommand\OUTPUT{\item[\algorithmicoutput]}

% Specify depth to show in then table of contents
% -1 part   1 section       3 subsubsection     5 subparagraph
% 0 chapter 2 subsection    4 paragraph
\setcounter{tocdepth}{4}

% Specify depth to show in document
\setcounter{secnumdepth}{3}

% Link color setup
\hypersetup{
	colorlinks = true,
	linkcolor = black,
	citecolor = blue
}

% Change format of page
\pagestyle{fancy}
\fancyhf{}
\fancyhead{}
\fancyfoot{}
\fancyhead[L]{Bài tập lớn - Cơ sở dữ liệu}
\fancyfoot[L]{Nhóm 8 - Ranking}
\fancyfoot[R]{\thepage}
\renewcommand{\headrulewidth}{1pt}
\renewcommand{\footrulewidth}{1pt}

% \patchcmd{\chapter}{\thispagestyle{plain}}{\thispagestyle{fancy}}{}{}

\renewcommand{\thesection}{\arabic{section}}
\renewcommand{\thesubsection}{\thesection.\arabic{subsection}}
\renewcommand{\thesubsubsection}{\thesubsection.\arabic{subsubsection}}

% format
\usepackage{titlesec}
\usepackage{etoolbox}
\makeatletter
\patchcmd{\ttlh@hang}{\parindent\z@}{\parindent\z@\leavevmode}{}{}
\patchcmd{\ttlh@hang}{\noindent}{}{}{}
\makeatother

\titleformat{\subsection}
{\normalfont\large\bfseries}{\thesubsection}{1em}{}
\titleformat{\subsubsection}
{\normalfont\large\sffamily\bfseries}{\thesubsubsection}{1em}{}

% tab command
\newcommand\tab[1][1cm]{\hspace*{#1}}

\begin{document}

%%%%%%%%%%%%%%%%%%%%%%%%% Title Page %%%%%%%%%%%%%%%%%%%%%%%%%%%%%
\thispagestyle{empty}
\thisfancypage{\setlength{\fboxsep}{0pt}\fbox}{}

\begin{center}
\begin{large}
TRƯỜNG ĐẠI HỌC BÁCH KHOA HÀ NỘI
\end{large} \\
\begin{large}
VIỆN CÔNG NGHỆ THÔNG TIN VÀ TRUYỀN THÔNG
\end{large} \\
\textbf{--------------------  *  ---------------------}\\[2.3cm]
{\fontsize{32pt}{1}\selectfont BÁO CÁO MÔN HỌC}\\
{\fontsize{20pt}{1}\selectfont CƠ SỞ DỮ LIỆU}\\[0.5cm]
{\fontsize{18pt}{1}\selectfont Tên đề tài: Độ phức tạp tính toán \\
    trong đa dạng hóa kết quả truy vấn}\\[0.5cm]
\includegraphics[width=4cm]{hust.jpg}\\[1.3cm]
\end{center}

\hspace{2.5cm} Nhóm sinh viên thực hiện : \hspace{4pt}
\textbf{\parbox[t]{5cm}{    
    Tạ Quang Tùng \\
    Phạm Minh Tâm\\
    Trần Bảo Ngọc
}}\\[12pt]

\hspace{2.5cm} Giáo viên hướng dẫn \hspace{24pt} :  \hspace{4pt} \textbf{\parbox[t]{5cm}{    
Nguyễn Kim Anh
}}

\vspace{1.5cm}
\begin{center}
{\fontsize{14pt}{1}\selectfont HÀ NỘI}\\
{\fontsize{14pt}{1}\selectfont \today}
\end{center}
\pagenumbering{gobble}
\tableofcontents 
\newpage

\pagenumbering{arabic}
\newpage
\setcounter{page}{1}

\section*{Lời nói đầu}
% To show section* in the table of contents
% Doesn't need with section
\addcontentsline{toc}{section}{Lời nói đầu}

Nhu cầu tìm kiếm thông tin của người dùng hiện nay ngày càng lớn và phức tạp. Thông thường người dùng thể hiện thông tin mình muốn tìm kiếm thông qua một số từ khóa và các từ khóa này thường không rõ ràng. Công cụ tìm kiếm phải đoán các kết có nhiều khả năng phù hợp nhất để đáp ứng nhu cầu khác nhau của người dùng. Đối với một truy vấn không cụ thể, công cụ tìm kiếm có thể tiếp cận theo nguyên tắc xếp hạng và trả lại các các kết quả có xác suất thỏa mãn nhu cầu người dùng cao nhất. Nhưng có thể người dùng không thỏa mãn bất cứ kết quả trả về nào. Vì thế 	công cụ tìm kiếm cần đa dạng hóa kết quả truy vấn.\\
Đa dạng hóa bao gồm sự cân bằng giữa độ phù hợp của kết quả đối với câu truy vấn và độ khác nhau giữa các kết quả truy vấn. Do đó vấn đề đa dạng hóa kết quả truy vấn được xem như là vấn đề tối ưu hóa với hai tiêu chí là mức độ phù hợp của kết quả đối với câu truy vấn và độ khác nhau giữa các kết quả truy vấn. Đa dạng hóa được xem như kết hợp cả xếp hạng (đưa ra các kết quả phù hợp hơn ở vị trí cao hơn) và nhóm (nhóm các kết quả tương tự với nhau). Điều này dẫn đến sự phát triển của nhiều hàm mục tiêu và nhiều giải thuật khác nhau.
Cần tạo ra một tập các tiên đề cho hàm mục tiêu nhằm chọn ra hàm mục tiêu phù hợp nhất. Với mỗi hàm mục tiêu lại làm cho thuật toán có độ phức tạp khác nhau.

\newpage

\section{Tiên đề trong đa dạng hóa kết quả truy vấn và một số hàm mục tiêu}
Trong chương này, chúng ta sẽ đề cập đến cách tiếp cận 
theo hướng tiên đề trong đa dạng hóa kết quả truy vấn \cite{axiom-paper}. 
Mục đích để trợ giúp cho việc lựa chọn hàm mục tiêu và đồng thời 
ràng buộc kết quả bài toán.
Chúng ta sẽ xem xét những hàm đã được đề xuất thoả mãn những tính chất, đồng thời chỉ ra rằng không tồn tại hàm thỏa mãn tất cả những tính chất được đề xuất dưới đây. 
\subsection{Khái niệm cơ bản}
Xét một tập hợp các bản ghi $U=\{u_1, u_2, ..., u_n\}$, trong đó $n \geq 2$
và giải sử tồn tại một tập hữu hạn tất cả các câu truy vấn $Q$. 
Với một câu truy vấn $q \in Q$ và một số nguyên k, chúng ta muốn nhận được kết quả là tập con $S_k \subset U$ của tập các bản ghi ban đầu (hay của cơ sở dữ liệu). 
Hàm tương thích của mỗi một bản ghi được xác định bởi hàm 
$w: U \times Q \to \mathbf{R}^+$, bản ghi càng phù hợp với câu truy vấn 
thì sẽ có giá trị càng cao. Mục tiêu đa dạng hóa kết quả có thể được hiểu đơn giản là việc các kết quả trả về không được tương tự nhau. Dưới dạng biểu thức, ta có thể định nghĩa hàm khoảng cách $d: U \times U \to \mathbf{R}^+$ giữa các bản ghi, ở đó giá trị càng nhỏ thể hiện rằng hai bản ghi càng tương tự nhau. Đồng thời ta muốn hàm khoảng cách phải có tính chất phân biệt: Với hai bản ghi bất kì $u, v \in U$, $d(u, v) = 0$ khi và chỉ khi $u = v$, và tính chất đối xứng: $d(u, v) = d(v, u)$. Tuy nhiên, không nhất thiết hàm khoảng cách phải tạo thành một metric. 

Chúng ta ở đây chỉ quan tâm đến việc lựa chọn tập kết quả chứ không quan tâm đến vấn đề xếp hạng các bạn ghi. Nếu chúng ta đã có tập kết quả, chúng ta có thể sắp xếp kết quả cuối cùng theo thứ tự tương thích với câu truy vấn ban đầu. 

Dưới dạng toán học, hàm lựa chọn tập kết quả $f: 2^U \times Q \times w \times d \to \mathbf{R}$ có thể được hiểu là gán mỗi một điểm số cho từng các tập con của $U$ khi cho trước một câu truy vấn $q \in Q$, một hàm tương 
thích $w(\cdot)$ và một hàm khoảng cách $d(\cdot, \cdot)$.
Cố định $q, w(\cdot), d(\cdot, \cdot)$ và một số nguyên $k \in \mathbf{Z}^+ (k \geq 2)$
, mục tiêu là chọn một tập con $S_k \subseteq U$ của các bản ghi sao cho giá trị của hàm $f$ là lớn nhất. 
$$S^*_k = \argmax_{\substack{S_k \subseteq U \\ |S_k| = k}}
f(S_k, q, w(\cdot), d(\cdot, \cdot)) $$
Trong đó tất cả các đối số khác $S_k$ đều được cố định. 

Ta có thể có rất nhiều lựa chọn hàm mục tiêu $f$ với 
các hàm tương thích và hàm khoảng cách cho trước. 
Những hàm đó có sự đánh đổi giữa tính tương thích và tính 
tương tự theo những cách khác nhau. 
Do đó chung ta cần chỉ định ra các 
tiêu chuẩn để lựa chọn hàm mục tiêu tốt trong vô số các hàm mục tiêu đó.
Cách tiếp cận toán học được sự dụng 
phổ biến trong trường hợp này là đưa ra 
một hệ tiên đề mà được mong đợi trong các hệ thống hỗ trợ đa dạng hóa. 
Từ đó cung cấp một cơ sở để so sánh sự khác biệt giữa các hàm mục tiêu được chọn. 
\subsection{Những tiên đề trong đa dạng hóa kết quả truy vấn}
Chúng ta đề xuất hàm $f$ thỏa mãn tập các tiên đề dưới đây. Mỗi một tiên đề
đều khá là trực quan đối với  vấn đề đa dạng hóa kết quả. 
Thêm vào đó, chúng ta sẽ chỉ ra rằng bất kì một tập con thực sự của những tiên đề này là "cực đại", có nghĩa là không tồn tại hàm mục tiêu nào thỏa mãn tất cả những tiên đề dưới đây. 
Từ đó cung cấp một phương pháp tự nhiên cho việc lựa chọn 
giữa các hàm mục tiêu, khi mà 
một số tính chất là thiết yếu cho một hệ thống nào đó. 

Cố định $q, w(\cdot), d(\cdot, \cdot), k$ , $f$ và  
$S^*_k = \argmax_{S_k \subseteq U}
f(S_k, q, w(\cdot), d(\cdot, \cdot)) $. Ta có các tiên đề sau: 
\begin{enumerate}
    \item \textbf{Bất biến theo tỉ lệ}: Tính chất này chỉ định 
        rằng hàm mục tiêu không được phép bị ảnh hưởng khi mà thay đổi 
        đầu vào theo cùng một tỉ lệ. Một cách hình thức, 
        xét tập tối ưu $S^*_k$, chúng ta muốn $f$ vẫn thỏa mãn
        $S^*_k = \argmax_{S_k \subseteq U} 
        f (S_k, q, \alpha \cdot w(\cdot), \alpha \cdot d(\cdot, \cdot))$ 
        với bất kì một giá trị dương của $\alpha$.

    \item \textbf{Nhất quán}: Tính nhất quán nói rằng nếu làm cho 
        những bản ghi trong tập kết quả càng tương thích 
        và càng có tính đa dạng hơn
        và đồng thời làm cho các bạn ghi không phải kết quả ít tương thích,
        ít có tích đa dạng hơn thì kết quả của bài toán vẫn không 
        thay đổi. Một cách hình thức, với hai hàm bất kì $\alpha: U \to 
        \mathbf{R}^+$ và $\beta: U \times U \to \mathbf{R}^+$, 
        chúng ta thay đổi hàm tương thích và hàm khoảng cách
        như sau: 
        $$w(u) = 
        \begin{cases}
            w(u) + \alpha(u) \text{\hspace{1cm}} u \in S^*_k \\
            w(u) - \alpha(u) \text{\hspace{1cm}Trường hợp còn lại}
        \end{cases}
        $$
        $$d(u, v) = 
        \begin{cases}
            d(u, v) + \beta(u, v) \text{\hspace{1cm}} u \in S^*_k \\
            d(u, v) - \beta(u, v) \text{\hspace{1cm}Trường hợp còn lại}
        \end{cases}
        $$
        Thì $S^*_k$ vẫn là tập tối ưu của hàm mục tiêu $f$. 
    \item \textbf{Phong phú}. Tính phong phú nói rằng ta có thể đạt được 
        bất kì một tập nào đó là kết quả, nếu như lựa chọn đúng hàm tương 
        thích và hàm khoảng cách. Một cách hình thức:
        $$
        \forall k \geq 2, \exists w(\cdot), \exists d(\cdot, \cdot), 
        !\exists S^*_k = \argmax_{S_k \subseteq U}
        f(S_k, q, w(\cdot), d(\cdot, \cdot))
        $$

    \item \textbf{Ổn định}. Tính ổn định quy định những hàm mà kết quả 
        bài toán không thay đổi tùy ý với kích thước của tập kết quả. Hàm 
        $f$ phải thoải mãn $S^*_k \subset S^*_{k+1}$.

    \item \textbf{Độc lập với phần tử không tương thích}. Tiên đề này nói
        nói rằng điểm số của tập hợp không bị ảnh hưởng bởi những bản ghi 
        nằm ngoài tập đó. Cụ thể, với một tập $S$ bất kì, hàm $f$ 
        tại $S$: $f(S)$ sẽ độc lập với những giá trị sau: 
        \begin{itemize}
            \item $w(u)$ với mọi $u \notin S$.
            \item $d(u, v)$ với mọi $u, v \notin S$.  
        \end{itemize}

    \item \textbf{Đơn điệu}. Tính đợi điệu yêu cầu việc thêm bản ghi vào 
        một tập hợp bất kì không làm tăng điểm số của hàm mục tiêu đối với 
        tập đó. Cố định $w(\cdot), d(\cdot, \cdot), f \text{ và } S \subseteq U$. 
        Với mọi $x \in S$, ta có: 
        $$f(S \cup \{x\}) \geq f(S)$$

    \item \textbf{Độ mạnh của tính tương thích}. Tính chất này đảm bảo
        rằng không có hàm $f$ nào bỏ qua hàm tương thích. Một cách hình 
        thức, chúng ta cố định $w(\cdot), d(\cdot, \cdot), f$ và $S$, những tính chất 
        sau đây phải đúng với mọi giá trị $x \in S$:
        \begin{enumerate}
            \item Tồn tại số thực $\delta_0 > 0$ 
                và $a_0 > 0$ để mà những điều
                kiện dưới đây được thoải mãn sau khi đã thực hiện chỉnh 
                sửa sau: Sửa đổi giá trị của hàm tương thích trở thành hàm 
                $w'(\cdot)$ sao cho $w'(\cdot)$ 
                giống hệt $w(\cdot)$ ngoại trừ tại phần
                tử $x$, $w'(x) = a_0 > w(x)$. Khi đó, ta có: 
                $$
                f(S, w'(\cdot), d(\cdot, \cdot), k) =
                f(S, w(\cdot)A, d(\cdot, \cdot), k) + \delta_0
                $$
            \item Nếu $f(S \setminus {x}) < f(S)$ thì tồi tại số thực 
                $\delta_1 > 0$ và $a_1 > 0$ để mà những điều kiện sau vẫn 
                đúng: chỉnh sửa hàm tương thích $w(\cdot)$ để đạt được một 
                hàm mới $w'(\cdot)$ sao cho hàm $w'(\cdot)$ giống hệt hàm 
                cũ, ngoại trừ tại phần tử $x$, $w'(x) = a_1 < w(x)$. 
                Từ đó, ta có: 
                $$
                    f(S, w'(\cdot), d(\cdot, \cdot), k) = 
                    f(S, w(\cdot), d(\cdot, \cdot), k) - \delta_1
                $$
        \end{enumerate}

    \item \textbf{Độ mạnh của tính tương tự}. Tính chất này đảm bảo rằng 
        không có hàm $f$ nào bỏ qua hàm khoảng cách. Một cách hình thức, 
        nếu cố định $w(\cdot)$, $d(\cdot, \cdot)$, $f$ và $S$, thì những 
        tính chất sau đúng với mọi giá trị $x \in S$:
        \begin{enumerate}
            \item Tồn tại số thực $\delta_0 > 0$ 
                và $a_0 > 0$ để mà những điều
                kiện dưới đây được thoải mãn sau khi đã thực hiện chỉnh 
                sửa sau: tạo một hàm 
                $d'(\cdot, \cdot)$ từ hàm $d(\cdot, \cdot)$, 
                trong đó, ta tăng giá trị của 
                $d(x, u)$ tại một số vị trí 
                $u$ cần thiết nào đó sao cho 
                $\min_{u \in S} d(x, u) = b_0$. 
                Từ đó, ta có: 
                $$
                f(S, w(\cdot), d'(\cdot, \cdot), k) =
                f(S, w(\cdot)A, d(\cdot, \cdot), k) + \delta_0
                $$
            \item Nếu $f(S \setminus {x}) < f(S)$ thì tồi tại số thực 
                $\delta_1 > 0$ và $a_1 > 0$ để mà những điều kiện sau vẫn 
                đúng: chỉnh sửa hàm khoảng cách $d(\cdot, \cdot)$
                bằng cách tăng giá trị $d(x, u)$ tại một số 
                vị trí $u$ cần thiết nào đó để đảm bảo rằng 
                $\max_{u \in S} d(x, u) = b_1$. Gọi hàm được tạo ra là 
                $d'(\cdot, \cdot)$. Từ đó, ta có: 
                $$
                    f(S, w(\cdot), d'(\cdot, \cdot), k) = 
                    f(S, w(\cdot), d(\cdot, \cdot), k) - \delta_1
                $$
        \end{enumerate}
\end{enumerate}

Từ những tiên đề này, một câu hỏi được đặt ra là làm thế nào để mô tả 
một tập các hàm $f$ thỏa mãn những tiên đề này. Đáng ngạc nhiên là không 
thể tìm được hàm thỏa mãn được tất cả những tiên đề cùng một lúc. 

\textit{Định lý: Không hàm $f$ thỏa mãn tất cả những tiên đề đã được nêu ở trên.}

Định lý này ngụ ý rằng bất kì một tập con thực sự của tập các tiên đề 
trên là tối đa. Kết quả này cho phép chúng ta mô tả một cách tự nhiên một 
tập các hàm đa dạng hóa, và lựa chọn một  hàm cụ thể thỏa mãn tập con các 
tiên đề mà chúng ta mong muốn. 

\subsection{Hàm mục tiêu và thuật toán}
Từ kết quả không thể của định lý trên, chúng ta chỉ có thể hi vọng những 
hàm đa dạng hóa có thể thỏa mãn một tập con của các tiên đề. 
Chú ý rằng số lượng các hàm thoải mãn có thể khá lớn. Hơn nữa, đề xuất một 
hàm mục tiêu có thể không hữu dụng nếu không thể tìm được một thuật toán 
để tối ưu hàm mục tiêu đã chọn. Trong phần này, chúng ta sẽ xem xét 
ba hàm mục tiêu cụ thể, và đồng thời 
cung cấp thuật toán tối ưu hàm mục tiêu đó. 

\subsubsection{Hàm đa dạng hóa tổng lớn nhất (max-sum diversification)}
Một hàm mục tiêu thỏa mãn đồng thời hai tiêu chuẩn (tương thích và đa dạng)
, biểu diễn dưới dạng tổng của hàm tương thích và hàm khoảng cách. Cụ thể 
được định nghĩa như sau: 
\begin{equation}
\label{max-sum-diversification}
f(S) = (k - 1) \sum_{u \in S} w(u) + 2 \lambda \sum_{u, v \in S} d(u, v)
\end{equation}
Ở đó $|S| = k$, và $\lambda > 0$ là tham số xác định sự đánh đổi giữa tính 
tương thích và tính đa dạng. Để ý rằng chúng ta cần nhân thêm giá trị 
ở tổng bên trái để cân bằng hóa vì tổng bên phải có
$\frac{k(k - 1)}{2}$ phần tử trong khi đó tổng bên 
trái chỉ có $k$ phần tử.

\textit{Nhận xét 1}. Hàm mục tiêu thỏa mãn phương trình 
\ref{max-sum-diversification} thỏa mãn tất cả các tiên đề ngoại trừ tiên đề
về tính ổn định. 

Hàm mục tiêu này có thể được chuyển về hàm mục tiêu phân tán cơ sở 
(facility dispersion), được biết đến là bài toán phân tán tổng lớn nhất
(max-sum dispersion problem). 
Bài toán phân tán tổng lớn nhất là bài toán phân tán mà mục tiêu 
là tối đa hóa tổng của 
tất cả các cặp khoảng cách giữa những điểm trong một tập $S$.
Trong trường hợp này, nếu ta định nghĩa hàm khoảng cách: 
\begin{equation}
\label{new-distance-msd}
    d'(u, v) = w(u) + w(v) + 2\lambda d(u, v)
\end{equation}
Dễ thấy, $d'(\cdot, \cdot)$ là một metric nếu như $d(\cdot, \cdot)$ 
cũng là một metric. Hơn nữa, với các giá trị 
$S \subseteq U$  ($|S| = k$), ta có: 
$$
    \sum_{u, v \in S} d'(u, v) = 
    (k - 1) \sum_{u \in S} w(u) +
    2 \lambda \sum_{u, v \in S} d(u, v)
$$
Từ đó, phương trình \ref{max-sum-diversification} có thể được viết lại 
thành: 
$$
f(S) = \sum_{u, v \in S} d'(u, v)
$$
Đồng thời, đó cũng là mục tiêu của bài toán phân tán tổng lớn nhất 
đã được mô tả ở trên. Từ đó ta có thể giải được bài toán 
về hàm mục tiêu
đa dạng hóa tổng lớn nhất. Bài toán tối đa hóa giá trị hàm mục tiêu của 
phương trình \ref{max-sum-diversification} là NP khó, nhưng tồn tại một 
thuật giải xấp xỉ cho bài toán. 
Trong trường hợp là metric, chúng ta có thể sử dụng 
thuật toán \ref{alg:1} để giải bài 
toán đã đặt ra. 

\begin{algorithm}
    \caption{Thuật toán cho bài toán phân tán tổng lớn nhất}
    \label{alg:1}
    \begin{algorithmic}
        \INPUT Tập $U$, giá trị nguyên k 
        \OUTPUT Tập $S$ ($|S|=k$) sao cho giá trị của $f(S)$ là lớn nhất
        \State Khởi tạo $S = \varnothing$
        \For { $i \gets 1$ \textbf{to} $\floor{\frac{k}{2}}$ }
            \State Tìm $(u, v) = \argmax_{x, y \in U} d(x, y)$
            \State Tập $S = S \cup \{u, v\}$
            \State Xóa tất cả các cạnh mà gắn với $u$ hoặc $v$ 
        \EndFor
        \If { k là lẻ }
            \State Thêm một bản ghi bất kì vào $S$ 
        \EndIf
    \end{algorithmic}
\end{algorithm}

\subsubsection{Hàm đa dạng hóa lớn nhất - nhỏ nhất 
(max-min diversification)}
Một hàm mục tiêu thứ hai thỏa mãn đồng thời hai tiêu chuẩn 
(tương thích và đa dạng)
, biểu diễn dưới dạng giá trị nhỏ nhất của tổng hàm tương thích và hàm 
khoảng cách. Cụ thể được định nghĩa như sau: 
\begin{equation}
    \label{equation:max-min}
    f(S) = \min_{u \in S} w(u) + 
    \lambda \min_{u, v \in S} d(u, v)
\end{equation}
Ở đó $|S| = k$, và $\lambda > 0$ là tham số chỉ định sự đánh đổi giữa 
tính tương thích và tính đa dạng. 

\textit{Nhận xét 2}. Hàm mục tiêu đa dạng hóa được cho bởi phương trình
\ref{equation:max-min} thỏa mãn tất cả những tiên đề ngoại trừ tiên đề 
về tính nhất quán và tính ổn định.

Như lần trước, hàm mục tiêu này tương ứng với một hàm mục tiêu phân tán 
cơ sở, trong trường hợp này là bài toán phân tán lớn nhất - nhỏ nhất 
(max-min dispersion problem). Hàm mục tiêu cho bài toán phân tán 
lớn nhất - nhỏ nhất là: 
$g(P) = \min_{v_i, v_j \in P} d(v_i, v_j)$, và có thể suy ra là tương 
đương với phương trình \ref{equation:max-min} bằng cách đặt khoảng cách: 
\begin{equation}
    d'(u, v) = \frac{1}{2}(w(u) + w(v)) + \lambda d(u, v)
\end{equation}
Cũng tương tự, $d'(u, v)$ tạo thành một không gian metric. Hơn nữa, ta 
lại có: 
$$
\min_{u, v \in S} d'(u, v) = \min_{u \in S} w(u)  
+ \lambda \min_{u, v \in S} d(u, v) = f(S)
$$
Từ đó, ta có thể giải bài toán tìm cực đại hàm mục tiêu đa dạng hóa lớn 
nhất nhỏ nhất bằng việc giải bài toán phân tán lớn nhất - nhỏ nhất. 
Ta có thể sử dụng thuật toán \ref{alg:2} để giải bài toán đó. 
Đồng thời, bài toán đã đặt ra cũng là bài toán NP khó. 

\begin{algorithm}
    \caption{Thuật toán cho bài toán phân tán lớn nhất - nhỏ nhất}
    \label{alg:2}
    \begin{algorithmic}
        \INPUT Tập $U$, giá trị nguyên k 
        \OUTPUT Tập $S$ ($|S|=k$) sao cho giá trị của $f(S)$ là lớn nhất
        \State Khởi tạo $S = \varnothing$
        \State Tìm $(u, v) = \argmax_{x, y \in U} d(x, y)$ 
        và tập $S = \{ u, v\}$

        \For {$x \in U$}
            \State Định nghĩa $d(x, S) = \min_{u \in S} d(x, u)$
        \EndFor

        \While {$|S| < k$}
            \State Tìm $x \in U \setminus S$ sao cho
            $x = \argmax_{x \in U \setminus S} d(x, S)$
            \State Tập $S = S \cup \{ x \}$
        \EndWhile
    \end{algorithmic}
\end{algorithm}

\subsubsection{Hàm đơn mục tiêu (mono-objective formulation)}
Hàm mục tiêu cuối cùng không liên quan đến bài toán phân tán cơ bản nào 
cả và nó hợp giá trị tương thích và giá trị của hàm khoảng cách 
thành một giá trị đơn nhất cho mỗi bản ghi.
Cụ thể được định nghĩa như sau: 
\begin{equation}
    \label{equation:mono}
    f(S) = \sum_{u \in S} w'(u)
\end{equation}
Trong đó
$$
w'(u) = w(u) + \frac{\lambda}{|U| - 1}
\sum_{u \in U} d(u, v)
$$
Giá trị tham số $\lambda$ thể hiện sử đánh đổi giữa tính tương thích 
và tính đa dạng. 
Một cách trực quan, giá trị $w'(u)$ tính toán sự quan trọng một cách toàn bộ mỗi một bản ghi $u$. Đặc trưng tiên đề của hàm mục tiêu này như sau:

\textit{Nhận xét 3}. Hàm mục tiêu được cho trong phương trình 
\ref{equation:mono} thỏa mãn tất cả các tiên đề trừ tiên đề về tính 
nhất quán.

Đồng thời nhận thấy rằng ta có thể tối ưu hàm mục tiêu một cách chính xác bằng việc tính toán giá trị $w'(u)$ cho mỗi bản ghi $u \in U$ và rồi lựa chọn những bản ghi trong top k giá trị để cho vào tập kết quả. 

\newpage


\section{Độ phức tạp trong vấn đề đa dạng hóa kết quả tìm kiếm}
\cite{main-paper}
Đa dạng hóa kết quả try vấn là một vấn đề tối ưu hóa đa điều kiện cho việc xếp hạng kết quả truy  vấn. 
Cho bộ dữ liệu D, một truy vấn Q và một số nguyên k, yêu cầu đặt ra là tìm một tập có k phần tử từ kết quả truy vấn Q(D) sao cho các bộ trả về có sự liên quan nhất có thể tới câu truy vấn và đồng thời trả về kết quả đa dạng nhất cho mỗi bộ. Tập con của Q(D) được sắp xếp bởi một hàm mục tiêu được định nghĩa bởi tính liên quan của các kết quả với câu truy vấn và tính đa dạng của tập kết quả. Đa dạng hóa kết quả truy vấn có rất nhiều ứng dụng trong cơ sở dữ liệu, trích xuất thông tin và các hoạt động tìm kiếm. 
Có 3 vấn đề liên quan đến đa dạng hóa kết quả truy vấn được định nghĩa đó là:
  \begin{itemize}
  \item Xác định xem liệu có tồn tại tập k phần tử thỏa mãn một ràng buộc về khía cạnh độ liên quan và độ đa dạng đối với truy vấn (QRD).
  \item Xếp hạng một tập k phần tử cho trước (DRP).
  \item Đếm xem có bao nhiêu tập k phần tử thỏa mãn điều kiện cho trước (RDC).
  \end{itemize}
Chúng ta nghiên cứu vấn đề trên với nhiều ngôn ngữ truy vấn cho 3 hàm mục tiêu \textit{max-sum diversification, max-min diversification, mono-objective formulation}.
Với mỗi vấn đề ta sẽ dùng các hàm mục tiêu khác nhau để truy vấn. Với mỗi trường hợp, các thông số sẽ được thay đổi để nghiên cứu về độ phức tạp của bài toán và tìm yếu tố ảnh hưởng tới độ phức tạp của bài toán. 
\subsection{Các vấn đề liên quan đến đa dạng hóa kết quả truy vấn}
	\subsubsection{QRD  (\textit{The query result diversification problem}) }
	Cho một truy vấn Q, tập cơ sở dữ liệu D, một số nguyên k, một cận B, hàm mục tiêu F.Khi đó ta nói một bộ $ U \subseteq Q(D) $ là một tập ứng viên nếu nó có k phần tử $|U| = k$. Một tập ứng viên U là thỏa mãn cho bài toán (Q,D,k,F,B) nếu $F(U) \ge B$.\\
	Vấn đề $QRD(\mathcal{L}_Q,F(.))$ được trình bày như sau: \\
	INPUT: 	Một cơ sở dữ liệu D, một truy vấn $Q\in \mathcal{L}_Q$, một hàm mục tiêu F(.) một số thực B và một số nguyên $k\ge1$.\\
	OUTPUT: Trả lời câu hởi liệu có tồn tại một tập thỏa mãn bàn toán (Q,D,k,F,B) nêu trên không?
	\subsubsection{DRP  (\textit{The diversity ranking problem}) }
	Với một tập U được cho trước bởi người dùng hoặc hệ thống, chúng ta sẽ xác định xem liệu tập U đó có đạt được mục tiêu đa dạng hóa kết quả từ đó đánh giá kết quả có thỏa mãn với người dùng không.Đây là một vấn đề quyết định khác có thể gọi là xếp hạng tập kết quả cho trước. Ta xem xét một tập U và một số r. Ta nói hạng của U là r, ký hiệu $ rank(U) = r$ nếu tồn tại một mảng S có r-1 phần tử các tập ứng viên cho vấn đề (Q,D,k) sao cho:\\
	 \begin{itemize}
  	 \item Tất cả các tập thuộc mảng $s_i \in S $ đều thỏa mãn $F(s_i) > F(U)$
 	 \item Bất kỳ một tập $s_k \notin S $ đều có $F(U) \ge F(s_k)$
  	 \end{itemize}
	Vấn đề $DRP(\mathcal{L}_Q,F(.))$ được trình bày như sau: \\
	INPUT: 	Một cơ sở dữ liệu D, một truy vấn $Q\in \mathcal{L}_Q$, một hàm mục tiêu F(.) một bộ U nằm trong thuộc Q(D) với |U| = k.\\
	OUTPUT: Trả lời câu hởi liệu $rank(U) \le r$?
	\subsubsection{RDC  (\textit{The result diversity counting problem}) }
	Cho một mục tiêu B chúng ta muốn biết có bao nhiêu tập U thỏa mãn mục tiêu trên.\\
	Vấn đề $RDC(\mathcal{L}_Q,F(.))$ được trình bày như sau: \\
	INPUT: 	Một cơ sở dữ liệu D, một truy vấn $Q\in \mathcal{L}_Q$, một hàm mục tiêu F(.) và một số nguyên $k$ như ở trong vấn đề $QRD(\mathcal{L}_Q,F(.))$ \\
	OUTPUT: Trả lời câu hởi có bao nhiêu bộ thỏa mãn bàn toán (Q,D,k,F,B) ?
\subsection{Độ phức tạp của bài toán}
Chương này sẽ tìm hiểu về độ phức tạo của các vấn đề  $QRD(\mathcal{L}_Q,F(.))$, $DRP(\mathcal{L}_Q,F(.))$, $RDC(\mathcal{L}_Q,F(.))$ sử dụng các ngôn ngữ CQ,UCQ,$\exists FO^+$ và FO với các hàm mục tiêu F(.) là $F_{MS}(.)$  (max-sum diversification), $F_{MM}(.)$ (max-min diversification) và $F_{mono}(.)$ (mono-objective formulation).\\
Với mỗi trường hợp ta nghiên cứu:
	\begin{enumerate}
		\item Độ phức tạp hỗn hợp khi cả truy vấn và dữ liệu D là thay đổi.
		\item Độ phức tạp dữ liệu khi truy vấn Q được định nghĩa và cố định, dữ liệu D thay đổi, độ phức tạp được đánh giá trên một truy vấn cố định đối với nhiều loại dữ liệu đầu vào.
	\end{enumerate} 
\subsubsection{Độ phức tạp hỗn hợp}
\textbf{QRD}  (\textit{The query result diversification problem}) 
	\begin{enumerate}
		\item Khi hàm mục tiêu là max-sum hoặc max-min ngôn ngữ truy vấn có ảnh hưởng tới độ phức tạp của thuật toán, cụ thể: là NP-complete khi $\mathcal{L}_Q$ là CQ, UCQ, $\exists FO^+$, nhưng trở thành PSPACE-complete khi $\mathcal{L}_Q$ là FO. Trong khi sự hiện diện của UCQ và $\exists FO^+$ không làm cho thuật toán phức tạp hơn so với khi dùng ngôn ngữ CQ, thì sự hiện diện của FO làm phức tạp hóa thuật toán.
		\item Khi hàm mục tiêu là mono-objective thuật toán trở nên phức tạp hơn khi $\mathcal{L}_Q$ là CQ,UCQ, $\exists FO^+$ với độ phức tạp là PSPACE-complete giống như khi dùng ngôn ngữ FO.
	\end{enumerate}
	
	\textit{Định lý 1: Cho vấn đề $QRD(\mathcal{L}_Q,F(.))$, khi hàm mục tiêu là max-sum hoặc max-min, độ phức tạp hỗn hợp là:
		\begin{itemize}
  		\item NP-complete khi $\mathcal{L}_Q$ là  CQ, UCQ, $\exists FO^+$
  		\item PSPACE-complete khi $\mathcal{L}_Q$ là FO
  		\end{itemize}
  	Còn khi hàm mục tiêu là mono-objective độ phức tạp hỗn hợp là:
  	\begin{itemize}
  		\item PSPACE-complete khi $\mathcal{L}_Q$ là CQ, UCQ, $\exists FO^+$ hay FO.
  		\end{itemize}
	 }
\textbf{DRP}  (\textit{The diversity ranking problem})
	\begin{enumerate}
		\item Giống như $QRD(\mathcal{L}_Q,F(.))$ khi F là max-min objective hoặc max-sum objective sự khác nhau của các ngôn ngữ ảnh hưởng đến độ phức tạp của thuật toán. Khi F là mono-objective thay đổi ngôn ngữ không ảnh hưởng tới độ phức tạp
		\item Khi hàm mục tiêu là max-sum hoặc max-min $DRP(\mathcal{L}_Q,F(.))$ là coNP-complete cho ngôn ngữ CQ, UCQ.
	\end{enumerate}
	\textit{Định lý 2: Cho vấn đề $DRP(\mathcal{L}_Q,F(.))$, khi hàm mục tiêu là max-sum hoặc max-min, độ phức tạp hỗn hợp là:
		\begin{itemize}
  		\item coNP-complete khi $\mathcal{L}_Q$ là  CQ, UCQ, $\exists FO^+$
  		\item PSPACE-complete khi $\mathcal{L}_Q$ là FO
  		\end{itemize}
  	Còn khi hàm mục tiêu là mono-objective độ phức tạp hỗn hợp là:
  	\begin{itemize}
  		\item PSPACE-complete khi $\mathcal{L}_Q$ là CQ, UCQ, $\exists FO^+$ hay FO
  		\end{itemize}
	 }
	
\textbf{RDC}  (\textit{The result diversity counting problem})
Khi F là $F_{MS} (.)$ hoặc $F_{MM} (.)$, vấn đề trở nên khó đối với ngôn ngữ FO hơn là với CQ, UCQ, $\exists FO^+$. Ngược lại $F_mono (.)$ tác động dến sự phức tạp của thuật toán nhiều hơn so với ảnh hưởng của ngôn ngữ  $\mathcal{L}_Q$. Điều này tương tự như các vấn đề $QRD(\mathcal{L}_Q,F(.))$, $DRP(\mathcal{L}_Q,F(.))$ đã nêu ở trên. Các kết quả được xác minh bởi sự giảm bớt.\\
	\textit{Định lý 3: Cho vấn đề $RDC(\mathcal{L}_Q,F(.))$, khi hàm mục tiêu là max-		sum hoặc max-min, độ phức tạp hỗn hợp là:
		\begin{itemize}
  		\item \#.NP-complete khi $\mathcal{L}_Q$ là  CQ, UCQ, $\exists FO^+$
  		\item \#.PSPACE-complete khi $\mathcal{L}_Q$ là FO
  		\end{itemize}
  	Còn khi hàm mục tiêu là mono-objective độ phức tạp hỗn hợp là:
  	\begin{itemize}
  		\item \#.PSPACE-complete khi $\mathcal{L}_Q$ là CQ, UCQ, $\exists FO^+$ hay FO
  		\end{itemize}
	 }
\subsubsection{Độ phức tạp dữ liệu}
\textbf{QRD}  (\textit{The query result diversification problem}) 
Khi F là $F_{MS} (.)$  hoăc  $F_{MM} (.)$, cố định truy vấn Q không làm thay đổi độ phức tạp của vấn đề  $QRD(\mathcal{L}_Q,F(.))$ cho ngôn ngữ CQ, UCQ, $\exists FO^+$, vấn đề vẫn có độ phức tạp là NP-comlete. Ngược lại khi $\mathcal{L}_Q$ là FO và F là $F_{MS} (.)$ hoặc $F_{MM} (.)$ vấn đề đơn giản hơn với độ phức tạp là NP-comlete. Khi F là $F_{mono} (.)$ vấn đề cũng trở nên dễ giải quyết hơn.\\
	\textit{Định lý 4: Cho vấn đề $QRD(\mathcal{L}_Q,F(.))$, độ phức tạp dữ liệu là:
		\begin{itemize}
  		\item NP-complete khi F là  $F_{MS} (.)$ hoặc $F_{MM} (.)$
  		\item Trong PTIME khi F là $F_{mono} (.)$
  		\end{itemize}
  		với $\mathcal{L}_Q$ là CQ, UCQ, $\exists FO^+$ hoặc FO\\
	 }
	 \\
\textbf{DRP}  (\textit{The diversity ranking problem})
Giống như vấn đề $QRD(\mathcal{L}_Q,F(.))$, cố định truy vấn Q làm cho  $DRP(\mathcal{L}_Q,F(.))$ đơn giản hơn khi:
	\begin{enumerate}
		\item Hàm mục tiêu F là $F_{mono} (.)$ 
		\item Khi ngôn ngữ truy vấn $\mathcal{L}_Q$ là FO, và hàm mục tiêu F là $F_{MS} (.)$ hoặc $F_{MM} (.)$
	\end{enumerate}
Độ phức tạp dữ liệu vẫn tương tự như độ phức tạp hỗn hợp của khi LQ là CQ, UCQ hoặc $\exists FO^+$, và khi F là $F_MS (.)$ hoặc $F_{MM} (.)$.\\
	\textit{Định lý 5: Cho vấn đề $DRP(\mathcal{L}_Q,F(.))$, độ phức tạp dữ liệu là:
		\begin{itemize}
  		\item coNP-complete khi F là  $F_{MS} (.)$ hoặc $F_{MM} (.)$
  		\item Trong PTIME khi F là $F_{mono} (.)$
  		\end{itemize}
  		với $\mathcal{L}_Q$ là CQ, UCQ, $\exists FO^+$ hoặc FO
	 }
	 \\
\textbf{RDC}  (\textit{The result diversity counting problem})
	\textit{Định lý 6: Cho vấn đề $RDC(\mathcal{L}_Q,F(.))$, độ phức tạp dữ liệu là:\\
		\begin{itemize}
  		\item \#P-complete dưới sụ giảm thiểu khi F là  $F_{MS} (.)$ hoặc $F_{MM} (.)$
  		\item \#P-complete dưới sự tối giản đa thức Turing  khi F là $F_{mono} (.)$
  		\end{itemize}
  		với $\mathcal{L}_Q$ là CQ, UCQ, $\exists FO^+$ hoặc FO
	 }
	 \\
\textbf{Tổng kết} Từ những kết quả trên chúng ta thấy;
	\begin{enumerate}
		\item Cả ngôn ngữ truy vấn và hàm mục tiêu đều có tác động đến độ phức 			tạp hỗn hợp của những vấn đề trên. Cụ thể hơn: 
			\begin{itemize}
  			\item \#P-complete khi F là $F_{MS} (.)$ hoặc $F_{MM} (.)$, những vấn đề này đối với ngôn ngữ truy vấn FO có độ phức tạp phức tạp lớn hơn so với các ngôn ngữ CQ, UCQ và $\exists FO^+$.
  			\item \#P-complete  Đối với $F_{mono} (.)$ hàm mục tiêu là yếu tố chính chi phối sự phức tạp.
  			\end{itemize} 
		\item Độ phức tạp của dữ liệu không liên quan tới ngôn ngữ truy vấn.
	\end{enumerate}
\subsection{Một số trường hợp đặc biệt}
\textbf{Truy vấn nhận dạng}

	\textit{Hệ quả 1: Với truy vấn định danh, khi F là $F_{MS} (.)$ hoặc $F_{MM} (.)$, cả độ phức tạp hỗn hợp và độ phức tạp dữ liệu là:
		\begin{itemize}
  		\item NP-complete cho vấn đề $QRD(\mathcal{L}_Q,F(.))$
  		\item coNP-complete cho vấn đề $DRP(\mathcal{L}_Q,F(.))$ và 
  		\item \#P-complete cho vấn đề $RDC(\mathcal{L}_Q,F(.))$ dưới sụ giảm thiểu
  		\end{itemize}
  	Khi F là $F_mono (.)$ cả độ phức tạp hỗn hợp và độ phức tạp dữ liệu là:
  		\begin{itemize}
  		\item trong thời gian PTIME cho vấn đề $QRD(\mathcal{L}_Q,F(.))$
  		\item trong thời gian PTIME cho vấn đề $DRP(\mathcal{L}_Q,F(.))$
  		\item \#P-complete cho vấn đề $RDC(\mathcal{L}_Q,F(.))$ dưới sự tối giản đa thức Turing
  		\end{itemize}
	 }
\textbf{Khi $\lambda = 0 $} \\
Tiếp theo, chúng ta nghiên cứu tác động của hàm liên quan đến tính đa dạng đối với sự phức tạp của thuật toán. Khi $\lambda =0$ chỉ có hàm tính sự đa dạng $\delta _{rel} (.,.)$ được sử dụng trong hàm mục tiêu F(.). \\
	\textit{Hệ quả 2: Với $\lambda = 0 $, khi F là $F_{MS} (.)$ hoặc $F_{MM} (.)$, độ phức tạp hỗn hợp của những vấn đề $QRD(\mathcal{L}_Q,F(.))$, $DRC(\mathcal{L}_Q,F(.))$ và $RDC(\mathcal{L}_Q,F(.))$ giống với độ phức tạp ta đã đề cập đến trong định lý 1,2 và 3 nhưng độ phức tạp dữ liệu là:
		\begin{itemize}
  		\item trong thời gian PTIME cho vấn đề $QRD(\mathcal{L}_Q,F(.))$
  		\item trong thời gian PTIME cho vấn đề $DRP(\mathcal{L}_Q,F(.))$ và 
  		\item \#P-complete dưới sự tối giản đa thức Turing cho vấn đề $RDC(\mathcal{L}_Q,F(.))$ khi F(.) là $F_{MS} (.)$, trong FP khi  F(.) là $F_{MM} (.)$
  		\end{itemize}
  	Khi F là $F_{mono} (.)$ độ phức tạp hỗn hợp là:
  		\begin{itemize}
  		\item NP-complete cho vấn đề $QRD(\mathcal{L}_Q,F(.))$ khi $\mathcal{L}_Q$ là  CQ, UCQ hoặc $\exists FO^+$ và PSPACE-complete khi $\mathcal{L}_Q$ là FO.
  		\item coNP-complete cho vấn đề $DRP(\mathcal{L}_Q,F(.))$ khi $\mathcal{L}_Q$ là  CQ, UCQ hoặc $\exists FO^+$ và PSPACE-complete khi $\mathcal{L}_Q$ là FO.
  		\item \#.NP-complete cho vấn đề $RDC(\mathcal{L}_Q,F(.))$ khi $\mathcal{L}_Q$ là  CQ, UCQ hoặc $\exists FO^+$ và \#.PSPACE-complete khi $\mathcal{L}_Q$ là FO.
  		\end{itemize}
  	Độ phức tạp dữ liệu giống như trong các định lý 4,5,6.
	 }
	\\ 
\textbf{Khi $\lambda = 1 $} \\
Không giống như kết quả ở hệ quả 2, việc bỏ hàm tính sự liên quan $\delta _{rel} (.,.)$ khỏi F (·) không làm cho thuật toán dễ dàng hơn. Thực tế, cả độ phức tạp kết hợp và độ phức tạp dữ liệu của $QRD(\mathcal{L}_Q,F(.))$, $DRP(\mathcal{L}_Q,F(.))$ và $RDC(\mathcal{L}_Q,F(.))$ vẫn không đổi. Điều này chứng minh hàm tính sự đa dạng $\delta _{rel} (.,.)$ chi phối sự phức tạp của những vấn đề này.\\
	\textit{Định lý 7: Khi $\lambda = 1 $ độ phức tạp hỗn hợp giống như trong định lý 1,2 và 3 độ phức tạp dữ liệu giống như trong định lý 4,5,6 và không đổi với các vấn đề $QRD(\mathcal{L}_Q,F(.))$, $DRP(\mathcal{L}_Q,F(.))$, $RDC(\mathcal{L}_Q,F(.))$ khi hàm mục tiêu là $F_{MS} (.)$,$F_{MM} (.)$,$F_{mono} (.)$ với các ngôn ngữ CQ, UCQ, $\exists FO^+$ và FO\\
	 }
\textbf{Khi k là hằng số} \\
	\textit{Hệ quả 3: Với k cố định
		\begin{itemize}
  		\item Độ phức tạp hỗn hợp như trong định lý 1,2,3 với các vấn đề $QRD(\mathcal{L}_Q,F(.))$,$DRP(\mathcal{L}_Q,F(.))$ và $RDC(\mathcal{L}_Q,F(.))$
  		\item Độ phức tạp dữ liệu là :\\
  			{\begin{itemize}
  			\item PTIME với vấn đề $QRD(\mathcal{L}_Q,F(.))$
  			\item PTIME với vấn đề $DRP(\mathcal{L}_Q,F(.))$
  			\item FP với vấn đề $RDC(\mathcal{L}_Q,F(.))$
  			\end{itemize}}
  		\end{itemize}
  		không quan trọng hàm mục tiêu là $F_{MS} (.)$,$F_{MM} (.)$,$F_{mono} (.)$ và ngôn ngữ truy vấn là CQ, UCQ, $\exists FO^+$ và FO.\\
	 }
\textbf{Tổng kết} Từ những kết quả trên chúng ta thấy;
	\begin{itemize}
  		\item Ảnh hưởng của ngôn ngữ $\mathcal{L}_Q$: Khi F là $F_{MS} (.)$ hoặc $F_{MM} (.)$, các vấn đề $QRD(\mathcal{L}_Q,F(.))$, $DRP(\mathcal{L}_Q,F(.))$ và $RDC(\mathcal{L}_Q,F(.))$ độ phức tạp cho ngôn ngữ FO cao hơn so với các ngôn ngữ CQ, UCQ và $\exists FO^+$.Khi F là  $F_{mono} (.)$ , các giới hạn về độ phức tạp kết hợp của các vấn đề này vẫn không thay đổi khi LQ thay đổi (Theorems 1, 2 và 3). Ngược lại, đối với lớp các truy vấn nhận dạng, tất cả những vấn đề này trở nên đơn giản hơn. Ngôn ngữ truy vấn $\mathcal{L}_Q$ không ảnh hưởng đến độ phức tạp của các vấn đề này.
  		\item Ảnh hưởng của hàm tính độ đa dạng và hàm tính độ liên quan: Độ phức tạp của bài toán xuất phát từ hàm tính độ đa dạng trong hàm mục tiêu.
  		\item Ảnh hưởng của k: Khi k là hằng số cố định, độ phức tạp dữ liệu của $QRD(\mathcal{L}_Q,F(.))$, $DRP(\mathcal{L}_Q,F(.))$ và $RDC(\mathcal{L}_Q,F(.))$ trở nên đơn giản hơn, với hàm mục tiêu F là $F_{MS} (.)$ hoặc $F_{MM} (.)$ 
  	\end{itemize}

\textbf{Bảng 1}  Độ phức tạp hỗn hợp và độ phức tạp sữ liệu	\\
\begin{tabular}{|m{2cm}|m{2cm}|m{2.5cm}|m{2.5cm}|m{2.8cm}|}
\hline 
Hàm mục tiêu & Ngôn ngữ & \multicolumn{3}{|c|}{Vấn đề}\\ 
\hline
&&$QRD(\mathcal{L}_Q,F(.))$& $DRP(\mathcal{L}_Q,F(.))$ & $RDC(\mathcal{L}_Q,F(.))$ \\
\hline
&& \multicolumn{3}{|c|}{Độ phức tạp hỗn hợp} \\
\hline 
\multirow{2}{10em}{$F_{MS} (.)$ và \\$F_{MM} (.)$}& CQ, UCQ và $\exists FO^+$ & NP-complete & coNP-complete &\#.NP-complete \\
\cline{2-5}
&FO& PSPACE-complete & PSPACE-complete & \#.PSPACE-complete \\
\hline
$F_{mono} (.)$ & CQ, UCQ $\exists FO^+$ , FO & PSPACE-complete & PSPACE-complete & \#.PSPACE-complete \\
\hline

&& \multicolumn{3}{|c|}{Độ phức tạp dữ liệu} \\
\hline 
$F_{MS} (.)$ và $F_{MM} (.)$ & CQ, UCQ và $\exists FO^+$,FO & NP-complete & coNP-complete &\#P-complete \\
\hline
$F_{mono} (.)$ & CQ, UCQ $\exists FO^+$ , FO & PTIME & PTIME & \#P-complete \\
\hline


\end{tabular}


\textbf{Bảng 2} Các trường hợp đặc biệt 	\\
\begin{tabular}{|m{2cm}|m{2cm}|m{2.5cm}|m{2.5cm}|m{2.8cm}|}
\hline 
Điều kiện & Độ phức tạp & \multicolumn{3}{|c|}{Vấn đề}\\ 
\hline
&&$QRD(\mathcal{L}_Q,F(.))$& $DRP(\mathcal{L}_Q,F(.))$ & $RDC(\mathcal{L}_Q,F(.))$ \\
\hline
Truy vấn định danh F là $F_{mono} (.)$ &Độ phức tạp hỗn hợp & PTIME & PTIME &\#P-complete(Turing) \\
\hline
$\lambda = 0$ F là $F_{MS} (.)$  &Độ phức tạp dữ liệu & PTIME & PTIME &\#P-complete(Turing) \\
\hline
$\lambda = 0$ F là $F_{MM} (.)$  &Độ phức tạp dữ liệu &  PTIME & PTIME & FP \\
\hline
$\lambda = 0$ F là $F_{mono} (.)$, $\mathcal{L}_Q$ là CQ, UCQ $\exists FO^+$ &Độ phức tạp hỗn hợp &NP-complete & coNP-complete &\#NP-complete  \\
\hline
k là hằng số &Độ phức tạp dữ liệu & PTIME & PTIME &FP \\
\hline
\end{tabular}
\bibliography{report}{}
\bibliographystyle{plain}

\end{document}

