\documentclass[12pt]{report}
\usepackage{graphicx}
\usepackage[utf8]{vietnam}
\usepackage[left=3cm, right=3cm, top=3cm, bottom =3cm]{geometry}
\usepackage{pdfpages}
\usepackage{fancyhdr}
\usepackage{fancybox}
\usepackage{hyperref}
\usepackage{etoolbox}
\usepackage{float}
\usepackage{amsmath}
\usepackage{amssymb}
\usepackage{mathtools}
\usepackage{indentfirst}

\usepackage{algorithm}
\usepackage{algorithmicx}
\usepackage{algpseudocode}

% Add commands for ceil and floor
\DeclarePairedDelimiter\ceil{\lceil}{\rceil}
\DeclarePairedDelimiter\floor{\lfloor}{\rfloor}

% Add command for argmax
\DeclareMathOperator*{\argmax}{argmax}

% Change name of algorithm
\floatname{algorithm}{Thuật toán}

% set up Input, Output in the algorithm block
\algnewcommand\algorithmicinput{\textbf{Đầu vào:}}
\algnewcommand\INPUT{\item[\algorithmicinput]}
\algnewcommand\algorithmicoutput{\textbf{Đầu ra:}}
\algnewcommand\OUTPUT{\item[\algorithmicoutput]}


% \setcounter{tocdepth}{4}

% Link color setup
\hypersetup{
	colorlinks = true,
	linkcolor = black,
	citecolor = blue
}

% Change format of page
\pagestyle{fancy}
\fancyhf{}
\fancyhead{}
\fancyfoot{}
\fancyhead[L]{Bài tập lớn - Cơ sở dữ liệu}
\fancyfoot[L]{Nhóm 8 - Ranking}
\fancyfoot[R]{\thepage}
\renewcommand{\headrulewidth}{1pt}
\renewcommand{\footrulewidth}{1pt}

% \patchcmd{\chapter}{\thispagestyle{plain}}{\thispagestyle{fancy}}{}{}

\renewcommand{\thesection}{\arabic{section}}
\renewcommand{\thesubsection}{\thesection.\arabic{subsection}}
\renewcommand{\thesubsubsection}{\thesubsection.\arabic{subsubsection}}

% format
\usepackage{titlesec}
\usepackage{etoolbox}
\makeatletter
\patchcmd{\ttlh@hang}{\parindent\z@}{\parindent\z@\leavevmode}{}{}
\patchcmd{\ttlh@hang}{\noindent}{}{}{}
\makeatother

\titleformat{\subsection}
{\normalfont\large\bfseries}{\thesubsection}{1em}{}
\titleformat{\subsubsection}
{\normalfont\large\sffamily\bfseries}{\thesubsubsection}{1em}{}

% tab command
\newcommand\tab[1][1cm]{\hspace*{#1}}

\begin{document}

%% Title Page %%%%%%%%%%%%%%%%%%%%%%%%%%%%%%%%%%%%%%%%%%%%%%%
%% ==> Write your text here or include other files.

\thispagestyle{empty}
\thisfancypage{\setlength{\fboxsep}{0pt}\fbox}{}

\begin{center}
\begin{large}
TRƯỜNG ĐẠI HỌC BÁCH KHOA HÀ NỘI
\end{large} \\
\begin{large}
VIỆN CÔNG NGHỆ THÔNG TIN VÀ TRUYỀN THÔNG
\end{large} \\
\textbf{--------------------  *  ---------------------}\\[2.3cm]
{\fontsize{32pt}{1}\selectfont BÁO CÁO MÔN HỌC}\\
{\fontsize{20pt}{1}\selectfont CƠ SỞ DỮ LIỆU}\\[0.5cm]
{\fontsize{18pt}{1}\selectfont Tên đề tài: Độ phức tạp tính toán \\
    trong đa dạng hóa kết quả truy vấn}\\[0.5cm]
\includegraphics[width=4cm]{hust.jpg}\\[1.3cm]
\end{center}

\hspace{2.5cm} Mhóm sinh viên thực hiện : \hspace{4pt}
\textbf{\parbox[t]{5cm}{    
    Tạ Quang Tùng \\
    Phạm Minh Tâm
}}\\[12pt]

\hspace{2.5cm} Giáo viên hướng dẫn \hspace{24pt} :  \hspace{4pt} \textbf{\parbox[t]{5cm}{    
Nguyễn Kim Anh
}}

\vspace{1.5cm}
\begin{center}
{\fontsize{14pt}{1}\selectfont HÀ NỘI}\\
{\fontsize{14pt}{1}\selectfont \today}
\end{center}
\pagenumbering{gobble}
\tableofcontents 
\newpage

\pagenumbering{arabic}
\newpage
\setcounter{page}{1}

\section*{Lời nói đầu}
% To show section* in the table of contents
% Doesn't need with section
\addcontentsline{toc}{section}{Lời nói đầu}
ABC
\section{Cách tiếp cận tiên đề trong đa dạng hóa kết quả truy vấn}
Trong chương này, chúng ta sẽ đề cập đến cách tiếp cận 
theo hướng tiên đề trong đa dạng hóa kết quả truy vấn \cite{diversification}. 
Mục đích để trợ giúp cho việc lựa chọn hàm mục tiêu và đồng thời 
ràng buộc kết quả bài toán.
Chúng ta sẽ xem xét những hàm đã được đề xuất thoả mãn những tính chất, đồng thời chỉ ra rằng không tồn tại hàm thỏa mãn tất cả những tính chất được đề xuất dưới đây. 
\subsection{Khái niệm cơ bản}
Xét một tập hợp các bản ghi $U=\{u_1, u_2, ..., u_n\}$, trong đó $n \geq 2$
và giải sử tồn tại một tập hữu hạn tất cả các câu truy vấn $Q$. 
Với một câu truy vấn $q \in Q$ và một số nguyên k, chúng ta muốn nhận được kết quả là tập con $S_k \subset U$ của tập các bản ghi ban đầu (hay của cơ sở dữ liệu). 
Hàm tương thích của mỗi một bản ghi được xác định bởi hàm 
$w: U \times Q \to \mathbf{R}^+$, bản ghi càng phù hợp với câu truy vấn 
thì sẽ có giá trị càng cao. Mục tiêu đa dạng hóa kết quả có thể được hiểu đơn giản là việc các kết quả trả về không được tương tự nhau. Dưới dạng biểu thức, ta có thể định nghĩa hàm khoảng cách $d: U \times U \to \mathbf{R}^+$ giữa các bản ghi, ở đó giá trị càng nhỏ thể hiện rằng hai bản ghi càng tương tự nhau. Đồng thời ta muốn hàm khoảng cách phải có tính chất phân biệt: Với hai bản ghi bất kì $u, v \in U$, $d(u, v) = 0$ khi và chỉ khi $u = v$, và tính chất đối xứng: $d(u, v) = d(v, u)$. Tuy nhiên, không nhất thiết hàm khoảng cách phải tạo thành một metric. 

Chúng ta ở đây chỉ quan tâm đến việc lựa chọn tập kết quả chứ không quan tâm đến vấn đề xếp hạng các bạn ghi. Nếu chúng ta đã có tập kết quả, chúng ta có thể sắp xếp kết quả cuối cùng theo thứ tự tương thích với câu truy vấn ban đầu. 

Dưới dạng toán học, hàm lựa chọn tập kết quả $f: 2^U \times Q \times w \times d \to \mathbf{R}$ có thể được hiểu là gán mỗi một điểm số cho từng các tập con của $U$ khi cho trước một câu truy vấn $q \in Q$, một hàm tương 
thích $w(\cdot)$ và một hàm khoảng cách $d(\cdot, \cdot)$.
Cố định $q, w(\cdot), d(\cdot, \cdot)$ và một số nguyên $k \in \mathbf{Z}^+ (k \geq 2)$
, mục tiêu là chọn một tập con $S_k \subseteq U$ của các bản ghi sao cho giá trị của hàm $f$ là lớn nhất. 
$$S^*_k = \argmax_{\substack{S_k \subseteq U \\ |S_k| = k}}
f(S_k, q, w(\cdot), d(\cdot, \cdot)) $$
Trong đó tất cả các đối số khác $S_k$ đều được cố định. 

Ta có thể có rất nhiều lựa chọn hàm mục tiêu $f$ với 
các hàm tương thích và hàm khoảng cách cho trước. 
Những hàm đó có sự đánh đổi giữa tính tương thích và tính 
tương tự theo những cách khác nhau. 
Do đó chung ta cần chỉ định ra các 
tiêu chuẩn để lựa chọn hàm mục tiêu tốt trong vô số các hàm mục tiêu đó.
Cách tiếp cận toán học được sự dụng 
phổ biến trong trường hợp này là đưa ra 
một hệ tiên đề mà được mong đợi trong các hệ thống hỗ trợ đa dạng hóa. 
Từ đó cung cấp một cơ sở để so sánh sự khác biệt giữa các hàm mục tiêu được chọn. 
\subsection{Những tiên đề trong đa dạng hóa kết quả truy vấn}
Chúng ta đề xuất hàm $f$ thỏa mãn tập các tiên đề dưới đây. Mỗi một tiên đề
đều khá là trực quan đối với  vấn đề đa dạng hóa kết quả. 
Thêm vào đó, chúng ta sẽ chỉ ra rằng bất kì một tập con thực sự của những tiên đề này là "cực đại", có nghĩa là không tồn tại hàm mục tiêu nào thỏa mãn tất cả những tiên đề dưới đây. 
Từ đó cung cấp một phương pháp tự nhiên cho việc lựa chọn 
giữa các hàm mục tiêu, khi mà 
một số tính chất là thiết yếu cho một hệ thống nào đó. 

Cố định $q, w(\cdot), d(\cdot, \cdot), k$ , $f$ và  
$S^*_k = \argmax_{S_k \subseteq U}
f(S_k, q, w(\cdot), d(\cdot, \cdot)) $. Ta có các tiên đề sau: 
\begin{enumerate}
    \item \textbf{Bất biến theo tỉ lệ}: Tính chất này chỉ định 
        rằng hàm mục tiêu không được phép bị ảnh hưởng khi mà thay đổi 
        đầu vào theo cùng một tỉ lệ. Một cách hình thức, 
        xét tập tối ưu $S^*_k$, chúng ta muốn $f$ vẫn thỏa mãn
        $S^*_k = \argmax_{S_k \subseteq U} 
        f (S_k, q, \alpha \cdot w(\cdot), \alpha \cdot d(\cdot, \cdot))$ 
        với bất kì một giá trị dương của $\alpha$.

    \item \textbf{Nhất quán}: Tính nhất quán nói rằng nếu làm cho 
        những bản ghi trong tập kết quả càng tương thích 
        và càng có tính đa dạng hơn
        và đồng thời làm cho các bạn ghi không phải kết quả ít tương thích,
        ít có tích đa dạng hơn thì kết quả của bài toán vẫn không 
        thay đổi. Một cách hình thức, với hai hàm bất kì $\alpha: U \to 
        \mathbf{R}^+$ và $\beta: U \times U \to \mathbf{R}^+$, 
        chúng ta thay đổi hàm tương thích và hàm khoảng cách
        như sau: 
        $$w(u) = 
        \begin{cases}
            w(u) + \alpha(u) \text{\hspace{1cm}} u \in S^*_k \\
            w(u) - \alpha(u) \text{\hspace{1cm}Trường hợp còn lại}
        \end{cases}
        $$
        $$d(u, v) = 
        \begin{cases}
            d(u, v) + \beta(u, v) \text{\hspace{1cm}} u \in S^*_k \\
            d(u, v) - \beta(u, v) \text{\hspace{1cm}Trường hợp còn lại}
        \end{cases}
        $$
        Thì $S^*_k$ vẫn là tập tối ưu của hàm mục tiêu $f$. 
    \item \textbf{Phong phú}. Tính phong phú nói rằng ta có thể đạt được 
        bất kì một tập nào đó là kết quả, nếu như lựa chọn đúng hàm tương 
        thích và hàm khoảng cách. Một cách hình thức:
        $$
        \forall k \geq 2, \exists w(\cdot), \exists d(\cdot, \cdot), 
        !\exists S^*_k = \argmax_{S_k \subseteq U}
        f(S_k, q, w(\cdot), d(\cdot, \cdot))
        $$

    \item \textbf{Ổn định}. Tính ổn định quy định những hàm mà kết quả 
        bài toán không thay đổi tùy ý với kích thước của tập kết quả. Hàm 
        $f$ phải thoải mãn $S^*_k \subset S^*_{k+1}$.

    \item \textbf{Độc lập với phần tử không tương thích}. Tiên đề này nói
        nói rằng điểm số của tập hợp không bị ảnh hưởng bởi những bản ghi 
        nằm ngoài tập đó. Cụ thể, với một tập $S$ bất kì, hàm $f$ 
        tại $S$: $f(S)$ sẽ độc lập với những giá trị sau: 
        \begin{itemize}
            \item $w(u)$ với mọi $u \notin S$.
            \item $d(u, v)$ với mọi $u, v \notin S$.  
        \end{itemize}

    \item \textbf{Đơn điệu}. Tính đợi điệu yêu cầu việc thêm bản ghi vào 
        một tập hợp bất kì không làm tăng điểm số của hàm mục tiêu đối với 
        tập đó. Cố định $w(\cdot), d(\cdot, \cdot), f \text{ và } S \subseteq U$. 
        Với mọi $x \in S$, ta có: 
        $$f(S \cup \{x\}) \geq f(S)$$

    \item \textbf{Độ mạnh của tính tương thích}. Tính chất này đảm bảo
        rằng không có hàm $f$ nào bỏ qua hàm tương thích. Một cách hình 
        thức, chúng ta cố định $w(\cdot), d(\cdot, \cdot), f$ và $S$, những tính chất 
        sau đây phải đúng với mọi giá trị $x \in S$:
        \begin{enumerate}
            \item Tồn tại số thực $\delta_0 > 0$ 
                và $a_0 > 0$ để mà những điều
                kiện dưới đây được thoải mãn sau khi đã thực hiện chỉnh 
                sửa sau: Sửa đổi giá trị của hàm tương thích trở thành hàm 
                $w'(\cdot)$ sao cho $w'(\cdot)$ 
                giống hệt $w(\cdot)$ ngoại trừ tại phần
                tử $x$, $w'(x) = a_0 > w(x)$. Khi đó, ta có: 
                $$
                f(S, w'(\cdot), d(\cdot, \cdot), k) =
                f(S, w(\cdot)A, d(\cdot, \cdot), k) + \delta_0
                $$
            \item Nếu $f(S \setminus {x}) < f(S)$ thì tồi tại số thực 
                $\delta_1 > 0$ và $a_1 > 0$ để mà những điều kiện sau vẫn 
                đúng: chỉnh sửa hàm tương thích $w(\cdot)$ để đạt được một 
                hàm mới $w'(\cdot)$ sao cho hàm $w'(\cdot)$ giống hệt hàm 
                cũ, ngoại trừ tại phần tử $x$, $w'(x) = a_1 < w(x)$. 
                Từ đó, ta có: 
                $$
                    f(S, w'(\cdot), d(\cdot, \cdot), k) = 
                    f(S, w(\cdot), d(\cdot, \cdot), k) - \delta_1
                $$
        \end{enumerate}

    \item \textbf{Độ mạnh của tính tương tự}. Tính chất này đảm bảo rằng 
        không có hàm $f$ nào bỏ qua hàm khoảng cách. Một cách hình thức, 
        nếu cố định $w(\cdot)$, $d(\cdot, \cdot)$, $f$ và $S$, thì những 
        tính chất sau đúng với mọi giá trị $x \in S$:
        \begin{enumerate}
            \item Tồn tại số thực $\delta_0 > 0$ 
                và $a_0 > 0$ để mà những điều
                kiện dưới đây được thoải mãn sau khi đã thực hiện chỉnh 
                sửa sau: tạo một hàm 
                $d'(\cdot, \cdot)$ từ hàm $d(\cdot, \cdot)$, 
                trong đó, ta tăng giá trị của 
                $d(x, u)$ tại một số vị trí 
                $u$ cần thiết nào đó sao cho 
                $\min_{u \in S} d(x, u) = b_0$. 
                Từ đó, ta có: 
                $$
                f(S, w(\cdot), d'(\cdot, \cdot), k) =
                f(S, w(\cdot)A, d(\cdot, \cdot), k) + \delta_0
                $$
            \item Nếu $f(S \setminus {x}) < f(S)$ thì tồi tại số thực 
                $\delta_1 > 0$ và $a_1 > 0$ để mà những điều kiện sau vẫn 
                đúng: chỉnh sửa hàm khoảng cách $d(\cdot, \cdot)$
                bằng cách tăng giá trị $d(x, u)$ tại một số 
                vị trí $u$ cần thiết nào đó để đảm bảo rằng 
                $\max_{u \in S} d(x, u) = b_1$. Gọi hàm được tạo ra là 
                $d'(\cdot, \cdot)$. Từ đó, ta có: 
                $$
                    f(S, w(\cdot), d'(\cdot, \cdot), k) = 
                    f(S, w(\cdot), d(\cdot, \cdot), k) - \delta_1
                $$
        \end{enumerate}
\end{enumerate}

Từ những tiên đề này, một câu hỏi được đặt ra là làm thế nào để mô tả 
một tập các hàm $f$ thỏa mãn những tiên đề này. Đáng ngạc nhiên là không 
thể tìm được hàm thỏa mãn được tất cả những tiên đề cùng một lúc. 

\textit{Định lý: Không hàm $f$ thỏa mãn tất cả những tiên đề đã được nêu ở trên.}

Định lý này ngụ ý rằng bất kì một tập con thực sự của tập các tiên đề 
trên là tối đa. Kết quả này cho phép chúng ta mô tả một cách tự nhiên một 
tập các hàm đa dạng hóa, và lựa chọn một  hàm cụ thể thỏa mãn tập con các 
tiên đề mà chúng ta mong muốn. 

\subsection{Hàm mục tiêu và thuật toán}
Từ kết quả không thể của định lý trên, chúng ta chỉ có thể hi vọng những 
hàm đa dạng hóa có thể thỏa mãn một tập con của các tiên đề. 
Chú ý rằng số lượng các hàm thoải mãn có thể khá lớn. Hơn nữa, đề xuất một 
hàm mục tiêu có thể không hữu dụng nếu không thể tìm được một thuật toán 
để tối ưu hàm mục tiêu đã chọn. Trong phần này, chúng ta sẽ xem xét 
ba hàm mục tiêu cụ thể, và đồng thời 
cung cấp thuật toán tối ưu hàm mục tiêu đó. 
\subsubsection{Hàm đa dạng hóa tổng lớn nhất (max-sum diversification)}
Một hàm mục tiêu thỏa mãn đồng thời hai tiêu chuẩn (tương thích và đa dạng)
, biểu diễn dưới dạng tổng của hàm tương thích và hàm khoảng cách. Cụ thể 
được định nghĩa như sau: 
\begin{equation}
\label{max-sum-diversification}
f(S) = (k - 1) \sum_{u \in S} w(u) + 2 \lambda \sum_{u, v \in S} d(u, v)
\end{equation}
Ở đó $|S| = k$, và $\lambda > 0$ là tham số xác định sự đánh đổi giữa tính 
tương thích và tính đa dạng. Để ý rằng chúng ta cần nhân thêm giá trị 
ở tổng bên trái để cân bằng hóa vì tổng bên phải có
$\frac{k(k - 1)}{2}$ phần tử trong khi đó tổng bên 
trái chỉ có $k$ phần tử.

\textit{Nhận xét}. Hàm mục tiêu thỏa mãn phương trình 
\ref{max-sum-diversification} thỏa mãn tất cả các tiên đề ngoại trừ tiên đề
về tính ổn định. 

Hàm mục tiêu này có thể được chuyển về hàm mục tiêu phân tán cơ sở 
(facility dispersion), được biết đến là bài toán phân tán tổng lớn nhất
(max-sum dispersion problem). 
Bài toán phân tán tổng lớn nhất là bài toán phân tán mà mục tiêu 
là tối đa hóa tổng của 
tất cả các cặp khoảng cách giữa những điểm trong một tập $S$.
Trong trường hợp này, nếu ta định nghĩa hàm khoảng cách: 
\begin{equation}
\label{new-distance-msd}
    d'(u, v) = w(u) + w(v) + 2\lambda d(u, v)
\end{equation}
Dễ thấy, $d'(\cdot, \cdot)$ là một metric nếu như $d(\cdot, \cdot)$ 
cũng là một metric. Hơn nữa, với các giá trị 
$S \subseteq U$  ($|S| = k$), ta có: 
$$
    \sum_{u, v \in S} d'(u, v) = 
    (k - 1) \sum_{u \in S} w(u) +
    2 \lambda \sum_{u, v \in S} d(u, v)
$$
Từ đó, phương trình \ref{max-sum-diversification} có thể được viết lại 
thành: 
$$
f(S) = \sum_{u, v \in S} d'(u, v)
$$
Đồng thời, đó cũng là mục tiêu của bài toán phân tán tổng lớn nhất 
đã được mô tả ở trên. Từ đó ta có thể giải được bài toán 
về hàm mục tiêu
đa dạng hóa tổng lớn nhất. Bài toán tối đa hóa giá trị hàm mục tiêu của 
phương trình \ref{max-sum-diversification} là NP khó, nhưng tồn tại một 
thuật giải xấp xỉ cho bài toán. 
Trong trường hợp là metric, chúng ta có thể sử dụng 
thuật toán \ref{alg:1} để giải bài 
toán đã đặt ra. 

\begin{algorithm}
    \caption{Thuật toán cho bài toán phân tán tổng lớn nhất}
    \label{alg:1}
    \begin{algorithmic}
        \INPUT Tập $U$, giá trị nguyên k 
        \OUTPUT Tập $S$ ($|S|=k$) sao cho giá trị của $f(S)$ là lớn nhất
        \State Khởi tạo $S = \varnothing$
        \For { $i \gets 1$ \textbf{to} $\floor{\frac{k}{2}}$ }
            \State Tìm $(u, v) = \argmax_{x, y \in U} d(x, y)$
            \State Tập $S = S \cup \{u, v\}$
            \State Xóa tất cả các cạnh mà gắn với $u$ và $v$ 
        \EndFor
        \If { k là lẻ }
            \State Thêm một bản ghi bất kì vào $S$ 
        \EndIf
    \end{algorithmic}
\end{algorithm}

\bibliography{report}{}
\bibliographystyle{plain}

\end{document}
